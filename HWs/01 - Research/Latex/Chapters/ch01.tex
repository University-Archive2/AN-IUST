عنوان مقاله‌ی استفاده شده، 
\lr{Mobile Edge Computing: A Survey}
است که در 
\href{https://ieeexplore.ieee.org/document/8030322}{این پیوند}
زیر اطلاعات بیشتری در مورد این مقاله موجود است.

این مقاله، تکنولوژی نوظهور رایانش لبه‌ی موبایل 
\LTRfootnote{\lr{Mobile Edge Computing (MEC})}
را معرفی می‌کند. در ادامه تحقیقات و پیشرفت‌های صورت گرفته در این حوزه و همچنین معماری‌های مختلف و مزایای آن را معرفی می‌کند. 

رایانش لبه‌ی موبایل، تکنولوژی را به کاربر نهایی نزدیک‌تر می‌کند و امکانات رایانش ابری 
\LTRfootnote{\lr{Cloud Computing}}
را در شبکه‌ی دسترسی رادیویی 
\LTRfootnote{\lr{Radio Access Netword (RAN)}}
در اختیار کاربران قرار می‌دهد که باعث کاهش تاخیر 
\LTRfootnote{\lr{Latency}}
و مصرف انرژی خواهد شد. شبکه‌ی دسترسی رادیویی ارتباط میان دستگاه‌های تلفن همراه و هسته‌ی شبکه را تسهیل می‌کند. 

از طرف دیگر، تکنولوژی قدیمی‌ترِ رایانش ابری موبایل، به علت وجود سرورهای ابری متمرکز، فاصله‌ی کاربران نهایی تا فضای ابری بسیار دور است که می‌تواند تاخیر را زیاد و ارتباط را مختل کند. 

کاربرهای مختلفی هم برای رایانش لبه‌ی موبایل ذکر شده که واقعیت افزوده 
\LTRfootnote{\lr{Augmented Reality (AR)}}
و حافظه‌های موقت 
\LTRfootnote{\lr{Cache}}
برای محتوا نمونه‌هایی از آن هستند. البته این تکنولوژی کاربرهای جدید دیگری مانند تحلیل ویدیو (مثلا در ترافیک)، خودروهای متصل به یکدیگر و... نیز پیدا کرده است که نوظهور هستند.
